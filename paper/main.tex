% THIS IS AN EXAMPLE DOCUMENT FOR VLDB 2012
% based on ACM SIGPROC-SP.TEX VERSION 2.7
% Modified by  Gerald Weber <gerald@cs.auckland.ac.nz>
% Removed the requirement to include *bbl file in here. (AhmetSacan, Sep2012)
% Fixed the equation on page 3 to prevent line overflow. (AhmetSacan, Sep2012)

\documentclass[sigconf]{acmart}

\settopmatter{printacmref=false} % Removes citation information below abstract
\renewcommand\footnotetextcopyrightpermission[1]{} % removes footnote with conference information in first column
%\pagestyle{plain} % removes running headers

\usepackage{booktabs} % For formal tables

\usepackage{url}
\usepackage{graphicx}
\usepackage{balance}  % for  \balance command ON LAST PAGE  (only there!)
\usepackage{xcolor}
\usepackage{subcaption}
\usepackage{caption}
\usepackage[ruled,vlined,linesnumbered]{algorithm2e}
\usepackage[noend]{algpseudocode}
\usepackage{paralist}
\usepackage{listings}
\usepackage{fancyvrb}

\lstset{basicstyle=\ttfamily,
escapeinside={||},
mathescape=true}

%\newcommand\todo[1]{\textcolor{red}{#1}}
%\newcommand\srm[1]{}
\newcommand\srm[1]{\textcolor{red}{SAM:#1}}
\newcommand\ani[1]{\textcolor{blue}{ANI:#1}}


\newcommand{\todo}[1]{{\leavevmode\color{red}#1}}

\setcopyright{rightsretained}

%Conference
\acmConference[SIGMOD'18]{ACM SIGMOD/PODS International Conference on Management of Data}{June 2018}{Houston, Texas, U.S.A}
\acmYear{2018}
\copyrightyear{2018}

\begin{document}

\title{Analytics using GPU vs CPU: How different are they really?}

\begin{abstract}
There has been a significant excitement amount and recent work on gpu-based
database systems. Previous work have shown that these systems can perform an
order of magnitude better than cpu-based database system on analytical
workloads such as those found in data warehouses, decision support and business
intelligence applications. The elevator pitch behind this performance
difference is straightforward: gpu's have higher compute and memory bandwidth
than the cpu. 

A hardware guy would view this as hype. Given the general notion that database
operators are bandwidth-bound, one would expect the maximum gain would be the
ratio of the memory bandwidth of GPU to that of GPU. Also, given the
restrictive GPU programming model which requires running program over a large
number of threads, one would expect additional materializations leading to
further degradation in gain. In this paper, we demonstrate that both these
assumptions are false. We first show that join and sort achieve speedup larger
than the bandwidth ratio. Second, using known GPU optimization techniques, it
is possible to avoid the 'paradigm' tax. Finally, it is known that GPUs are
more expensive that CPU. We do a performance/cost analysis to show that GPUs
can be around 3x more cost-efficient when the critical section of the
computation fits in GPU memory.
\end{abstract}

\maketitle

\section{Introduction}

There is a lot of excitement around using accelerators like GPU and FPGAs to
accelerate analytics. However, GPUs also have a lot of potential to accelerate
memory-bound applications like the ones we considered in NVL. There are two main
contributing factors for why GPU's are appealing now:

\begin{itemize}
\item The latest generation of GPU's have large amount of memory (the latest K80
Tesla card has 24GB memory, 3 years ago most papers had graphics card with at
max 4 GB memory) and significantly higher memory bandwidth than a CPU (CPU
bandwidth ~ 60 GBps compared to 480 GBps in K80). On a local machine with Titan
X GPU, we observed a memory bandwidth of 280 GBps on GPU (listed as max 330 GBps
in device spec) compared to 47GBps on CPU. The PCIe transfer speed is much
slower. As a result, all the works so far, which had to ship data from CPU to
GPU, suffered from limited gains as PCIe transfer time becomes the bottleneck.
The large memory allows us to keep/cache all or a good fraction of the dataset
on the GPU itself, eliminating the PCIe transfer overhead. 

\item GPUs are becoming commodities. Azure recently launched the N-series which are
machines with GPUs. The largest of them comes with 2 K80 cards having an
aggregate of 48GB GPU memory. The price point is $\$2.48ph$ for 24 cores/224GB
RAM in addition to 2 GPUs. This is comparable to $\$1.83$ for 20 core/140GB RAM
machine with no GPU. More cards can be stacked, MapD has machines with 8 K80
cards attached on the same machine. 
\end{itemize}

Existing work does apples-to-oranges comparison 1) fails to compare against
optimal implementation of both (applies to optimization to only one side) or
compares against systems known to be slow 2) fail to exploit the large memory of
modern GPU to cache data.


\input{background}
\section{Operator-level Comparison}

There has been considerable research on optimizing standard relational operators
for cpu-based and gpu-based databases. In this section, we compare the 
performance of running project, select, join, group-by and sort operations on the 
GPU versus running them on the CPU. In all cases, we assume that the data is
already in the device memory.

\subsection{Project}

We consider two forms of projection queries: one which involves linear combination 
of columns (Q1) and one involving user defined function (Q2) as shown below:

\begin{lstlisting}
Q1: SELECT a$x_1$ + b$x_2$ FROM t;
Q2: SELECT S(a$x_1$ + b$x_2$) FROM t;
\end{lstlisting}

$x_1$ and $x_2$ are 4-byte floating point values and $S$ is the sigmoid function ($S(x) = \frac{1}{1+e^{-x}}$).
\todo{ Talk about CPU SIMD. Vectorized execution. }
Figure~\ref{fig:projectres} shows the result.

\subsection{Select}

We now turn our attention to a micro-benchmark to test conjunctive selections:

\begin{lstlisting}
Q1: SELECT COUNT() FROM t WHERE $y_1$ < $v_1$ and $y_2$ < $v_2$
Q2: SELECT y_3 FROM t WHERE $y_1$ < $v_1$ and $y_2$ < $v_2$
\end{lstlisting}

Graph brbr/brnonbr/nonbrnonbr CPU/GPU varying sel Q1

Graph brbr/brnonbr/nonbrnonbr CPU/GPU varying sel Q2

$y_i$ is a decimal represented as a 4-byte integer. 

\subsection{Join}



\subsection{Group-By}


\subsection{Sort}


%\section{Cost Modelling}

It turns out Tensorflow does not have any cost model. Atleast there is no such
thing in the public release. User has to manually place computation on different
devices. There has been some work on cost modelling of relational operators for
GPU operators \cite{gpudb}. This was relatively easy due to the small number of
operators. For arbitrary operators, it can possibly be done but would require
much more work.

\subsection{Selection}

Symbols:

\begin{itemize}[label={},noitemsep]
\item $B_r$ - read bandwidth of global memory
\item $B_w$ - write bandwidth of global memory
\item $C_r$ - read segment size of global memory
\item $C_w$ - write segment size of global memory
\item $W$ - number of threads in the thread group
\item $\norm{R}$ - cardinality of table R
\item $n$ - number of projected columns
\item $K_i$ - attribute size of the ith projected column
\item $m$ - number of predicate columns
\item $P_i$ - the attribute size of the ith predicate columns
\item $r$ - selectivity of the predicates
\end{itemize}

\noindent \textbf{Old Model}: Scan and write out filter per column with predicate
\begin{align*}
T_1 &= \sum_{i=1}^{m} ( \ceil{\frac{P_i W}{C_r}} + \ceil{\frac{4 W}{C_r}} ) \times \frac{\norm{R}}{W} \times \frac{C_r}{B_r}  +  \sum_{i=1}^{m} \ceil{\frac{4 W}{C_w}} \times \frac{R}{W} \times \frac{C_w}{B_w}
\end{align*}
Read the filter and write results to global memory.
\begin{align*}
T_2 &= \sum_{i=1}^{n} (\ceil{\frac{4 W}{C_r}} + \ceil{\frac{K_i}{4}}) \times \frac{\norm{R}}{W} \times \frac{C_r}{B_r} + \norm{R} \times r \times \sum_{i=1}^{n} \ceil{\frac{K_i}{4}} \times \frac{C_w}{B_w}
\end{align*}

\noindent \textbf{New Model}: Scan all the columns and write out the filter
\begin{align*}
T_1 &= \sum_{i=1}^{m} \ceil{\frac{P_i W}{C_r}} \times \frac{\norm{R}}{W} \times \frac{C_r}{B_r}  +  \ceil{\frac{4 W}{C_w}} \times \frac{R}{W} \times \frac{C_w}{B_w}
\end{align*}
Read the filter and write results to global memory.
\begin{align*}
T_2 &= (\ceil{\frac{4 W}{C_r}} + \sum_{i=1}^{n} \ceil{\frac{K_i}{4}}) \times \frac{\norm{R}}{W} \times \frac{C_r}{B_r} + \sum_{i=1}^{n} \ceil{\frac{r R K_i}{C_w}} \times \frac{C_w}{B_w}
\end{align*}

\noindent \textbf{CPU Model}:

%\begin{center}
%\begin{tabu} to \textwidth { |X| X| X| }
 %\hline
 %Old Model & New Model & CPU Model \\ 
 %\hline




 %& cell5 

 %& cell6 \\  
 %\hline
 %cell7 & cell8 & cell9 \\ 
 %\hline
%\end{tabu}
%\end{center}


{
\bibliographystyle{ACM-Reference-Format}
\bibliography{bibs}  % vldb_sample.bib is the name of the Bibliography in this case
}

\input{appendix}

\end{document}

